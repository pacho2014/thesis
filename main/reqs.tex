
%-----------------------------------------------------------------------------
% Chapter: Introduction
%-----------------------------------------------------------------------------

\chapter{Requirements Analysis}
\label{chap:REQS}

The first step of this project is performing an overall requirements analysis
in order to determine the general needs of the \emph{Course2018} LMS.\ \ 
In this chapter, some key requirements will be investigated for their
feasibility and expectations.
Depending on the specifications of the requirements, they are categorized into
two categories, functional requirements and non-functional requirements.

\section{Terminologies}

\subsection{Functional requirement}
A functional requirement defines a task the system needs to accomplish,
as well as the behavior of the task, which is captured and described in detail
with its use cases.
These requirements will be further analyzed to design a system components
model~\cite{functionalReqs}.

\subsection{Non-functional requirement}
As opposed to functional requirements, non-functional requirements describe
non-behavioral specifications such as system performance, security, and system
environments.

\subsection{Use cases}
A use case is a series of actions performed by one or more actors interact
with the system to accomplished a task~\cite{useCase}.

\medskip 

Use cases in this thesis are composed of the following elements:

\begin{itemize}

\item \textbf{Primary actor}:
The main user (could be a person or an external system) who interacts with the
system.

\item \textbf{User story}:
A brief description of the task that the primary actor wishes to accomplish.

\item \textbf{Preconditions}:
Prerequisites to the test case;
conditions or states that must be satisfied prior to the start of the workflow.

\item \textbf{Postconditions}:
Conditions or states that must be satisfied after the workflow is finished.

\item \textbf{Triggers}:
An event (or a series of events) that starts the execution of the workflow.

\item \textbf{Basic flow}:
A step-by-step description of the detailed interactions between the actor and
the system, as well as the activities performed by the system.

\end{itemize}

\section{Functional requirements}
\label{FUNCREQS}

\subsection{User authentication}

The system shall provide routines to handle activities related to user accounts
such as user login, permission checking for restricted content, and user logout.

\subsubsection{Use cases}

\begin{enumerate}
\item User login:
\begin{itemize}
\item Primary actor:
    A registered user of the \emph{Course2018} LMS.
\item User story:
    A user wishes to have access to some restricted content, therefore, he
    first needs to login to the system.
\item Preconditions:
    The user is at the login interface, whose login credentials
    (namely, username and password) are input into the corresponding fields.
\item Postconditions:
    \begin{itemize}
        \item On success: an interface with the user's information (such as courses
            in which this user is registered) will be presented to the user.
        \item On failure: a login interface with an error message will be
            presented to the user.
    \end{itemize}
\item Triggers: the \emph{Login} button is clicked.
\item Basic flow:
    \begin{enumerate}
        \item The user's login credentials are encrypted and sent to the
            server.
        \item The server compares the user's login credentials against the
            account database.
        \item If a match is found in the database, an interface with the user's
            information will be returned and the user will have access to
            some restricted content to which he is authorized; otherwise, a
            login interface with error messages will be returned.
    \end{enumerate}
\end{itemize}

\item Permission checking:
\begin{itemize}
\item Primary actor: 
    A registered user of the \emph{Course2018} LMS
\item User story:
    The user request for accessing content for which he may or may not have
    permission.
\need 1 in
\item Preconditions:
    \begin{itemize}
        \item The user is logged into the system.
    \end{itemize}
\item Postconditions:
    \begin{itemize}
        \item On success: an interface with the requested content will be presented
            to the user.
        \item On failure: an interface indicates that the user is unauthorized will
            be presented to the user.
    \end{itemize}
\item Triggers: the user types the request URL to the browser location field
    or clicks on a link to the URL.
\item Basic flow:
    \begin{enumerate}
        \item The request (i.e., the desired URL) of the user is sent to the server.
        \item The server checks if the request is valid, namely, if the user
            is logged in and if the user has permission to access the requested
            content.
        \item If the request is valid, an interface with the requested content will
            be returned; otherwise, an interface with an error message 
            indicating the request is unauthorized will be returned.
    \end{enumerate}
\end{itemize}

\end{enumerate}


\subsection{Course information handling}
The system shall provide instructors with abilities to manage (i.e., create, remove,
and modify) their courses.
\subsubsection{Use cases}
\begin{enumerate}
\item Create a course:
\begin{itemize}
    \item Primary actor: a registered instructor.
    \item User story: the user wishes to create a course.
    \need 2 in
    \item Preconditions: 
        \begin{itemize}
            \item The user is logged in to the system.
            \item The user is at the create course interface where a form with
                course information related fields is presented to the user.
            \item The user fills in all the fields with data that may or
                may not be invalid.
        \end{itemize}
    \item Postconditions:
        \begin{itemize}
            \item On success: a summary interface of the course the user created
                will be presented.
            \item On failure: the same create course interface with error
                messages indicating which fields were filled with invalid data
                will be presented.
        \end{itemize}
    \item Triggers: the \emph{Create} button is clicked.
    \item Basic flow:
        \begin{enumerate}
            \item The form with the update course information is encrypted and
                sent to the server.
            \item The server validates the form data for errors including,
                but not limited to, invalid input format, numbers out-of-range,
                and conflicts (such as a course with the same identifier  
                already exists for that academic year).
            \item If the validation fails, the same create course interface
                with error messages will be returned; otherwise, the server will
                execute the routines to create a course with the submitted 
                information and have the summary interface of that course prepared
                and returned.
        \end{enumerate}
\end{itemize}

\item Update course information
\begin{itemize}
    \item Primary actor: a registered instructor.
    \item User story: the user wishes to update the information of one course.
    \item Preconditions:
        \begin{itemize}
            \item The user is logged in to the system.
            \item The user has previously created a course.
            \item The user is at the manage course interface for the course he
                wishes to modify, where a form with
                the course's existing information pre-filled is presented to
                him.
            \item The user has updated some fields with some new data that may
                or may not be invalid.
        \end{itemize}
    \item Postconditions:
        \begin{itemize}
            \item On success: a manage course interface where a form with
                the course's updated information pre-filled will be presented
                to the user.
            \item On failure: a manage course interface with error messages
                indicating which fields were filled with invalid data will be
                presented to the user.
        \end{itemize}
    \item Triggers: the \emph{Save Changes} button is clicked.
    \item Basic flow:
        \begin{enumerate}
            \item The form with course-related data is encrypted and sent to
                the server.
            \item The server validates the form data for errors including,
                but not limited to, invalid input format, numbers out-of-range,
                and conflicts (such as a course with the same identifier  
                already exists in that academic year).
            \item If the validation failed, the same create course interface
                with error messages will be returned; otherwise, the server will
                execute the routines to update the course with the submitted 
                information and have the new manage course interface with the
                updated course data prepared and returned.
        \end{enumerate}
\end{itemize}

\item Remove a course:
\begin{itemize}
    \item Primary actor: a registered instructor.
    \item User story: the user wishes to remove a course from the system.
    \item Preconditions:
        \begin{enumerate}
            \item The user is logged in to the system.
            \item The user has previously created a course.
            \item The user is at the manage course interface for the course
                he wishes to remove.
            \item The user clicked the \emph{Delete Course} button and a
                confirmation dialogue is presented to him.
        \end{enumerate}
    \item Postconditions:
        The course, along with all relevant data (such as registered students,
        assignments, and student grades) will be removed from the system
        database.
    \item Triggers: the \emph{Confirm} button on the confirmation dialogue is
        clicked.
    \item Basic flow:
        \begin{enumerate}
            \item A \emph{delete} request with the course identifier is encrypted
                and sent to the server.
            \item The server examines the request and makes sure the user is
                authorized to perform such an activity.
            \item If the request is valid, the course along with all its
                relevant data will be removed from the system database.
        \end{enumerate}
\end{itemize}

\end{enumerate}

\subsection{Assignment information handling}
The system shall provide the ability for instructors to manage (i.e., create,
remove, and modify) the assignments of their courses, as well as the ability for
students to submit and view those assignments.

\subsubsection{Use cases}
\begin{enumerate}
\item Create an assignment:
\begin{itemize}
    \item Primary actor: a registered instructor.
    \item User story: the user wishes to create an assignment.
    \item Preconditions:
        \begin{itemize}
            \item The user is logged in to the system.
            \item The user has previously created a course.
            \item The user is at the create assignment interface where a form
                with several assignment information-related fields is
                presented to the user.
            \item The user has filled in all the fields with data that may
                or may not be invalid.
        \end{itemize}
    \item Postconditions:
        \begin{itemize}
            \item On success: an summary interface of the assignment the user
                created will be presented.
            \item On failure: the same create assignment interface with error
                messages indicating which fields were filled with invalid data
                will be presented.
        \end{itemize}
    \item Triggers: the \emph{Create} button is clicked.
    \item Basic flow:
        \begin{enumerate}
            \item The form with the assignment-related data is encrypted and
                sent to the server.
            \item The server validates the form data for errors  including,
                but not limited to, invalid input format, numbers out-of-range,
                and conflicts (such as, an assignment with the same
                number already exists in the course).
            \item If the validation failed, the same create assignment interface
                with error messages will be returned; otherwise, the server will
                execute the routines to create an assignment with the submitted
                information and have the summary interface of that assignment
                prepared and returned.
        \end{enumerate}
\end{itemize}

\item Update assignment information:
\begin{itemize}
    \item Primary actor: a registered instructor.
    \item User story: the user wishes to update the information of one
        assignment.
    \item Preconditions:
        \begin{itemize}
            \item The user is logged in to the system.
            \item The user has previously created a course.
            \item The user has previously created an assignment for that course.
            \item The user is at the summary interface of the assignment he
                wishes to modify, where a form with the assignment's existing
                information pre-filled is presented to him.
        \end{itemize}
    \item Postconditions:
        \begin{itemize}
            \item On success: the same assignment summary interface with the updated
                information will be presented to the user.
            \item On failure: the same assignment summary interface with error
                messages indicating which fields were filled with invalid data will
                be presented to the user.
        \end{itemize}
    \item Triggers: the \emph{Save Changes} button is clicked.
    \item Basic flow:
        \begin{enumerate}
            \item The form with the updated assignment information is encrypted
                and sent to the server.
            \item The server validates the form data for the same errors
                mentioned in the previous use case.
            \item If the validation failed, the same assignment summary interface
                with error messages will be returned; otherwise, the server will
                execute the routines to update the assignment with the submitted
                information and have the new summary interface with the updated
                information prepared and returned.
        \end{enumerate}
\end{itemize}

\item Remove an assignment:
\begin{itemize}
    \item Primary actor: a registered instructor.
    \item User story: the user wishes to have one assignment removed.
    \item Preconditions:
        \begin{itemize}
            \item The user is logged in to the system.
            \item The user has previously created a course.
            \item The user has previously created an assignment for that course.
            \item The user is at the summary interface of the assignment he
                wishes to remove.
            \item The user clicked the \emph{Delete Assignment} button and a
                confirmation dialogue is presented to him.
        \end{itemize}
    \item Postconditions:
        The assignment, along with all the relevant data (such as assignment
        attachments, student grades and marker comments) will be removed from
        the system database.
    \item Triggers: the \emph{Confirm} button on the confirmation dialogue is
        clicked.
    \item Basic flow:
        \begin{enumerate}
            \item A \emph{delete} request with the assignment identifier is encrypted
                and sent to the server.
            \item The server examines the request and makes sure the user is
                authorized to perform such an activity.
            \item If the request is valid, the assignment, along with all its
                relevant data, will be removed from the system database.
        \end{enumerate}
\end{itemize}

\item View an assignment:
\begin{itemize}
    \item Primary actor: a registered student.
    \item User story: the user wishes to view the information of an assignment
     (such as assignment descriptions and attachments).
    \item Preconditions:
        \begin{itemize}
            \item The user is logged in to the system.
            \item The user is registered in a course whose instructor has
                prepared some assignments.
        \end{itemize}
    \item Postconditions: an interface with the assignment information the user
        wishes to view will be presented.
    \item Triggers: the user clicked a link to the assignment detail interface.
    \need 1 in
    \item Basic flow:
        \begin{enumerate}
            \item A request with the assignment identifier is encrypted and
                sent to the server.
            \item The server examines the request and makes sure the user has
                permission to access the requested content (i.e., the user is
                registered in the course to which the assignment belongs).
            \item If the request is valid, an interface with the requested
                content will be prepared and return.
        \end{enumerate}
\end{itemize}
\end{enumerate}

\subsection{File handling}
The system shall provide abilities for users to upload, view, and download
files when they need to (such as submit assignments and upload attachments).

\subsubsection{Use cases}
\begin{enumerate}

\item Upload files:
\begin{itemize}
    \item Primary actor: a registered student.
    \item User story: the user wishes to submit an assignment.
    \item Preconditions:
        \begin{itemize}
            \item The user is logged in to the system.
            \item The user is registered in the course for which he wishes to
                submit an assignment.
            \item The user is at the submit assignment interface where an
                \emph{Upload} button is presented to him.
        \end{itemize}
    \item Postconditions:
        \begin{itemize}
            \item The files the user submitted will be uploaded and saved to the
                \emph{Course2018}'s file system.
            \item Theses files can be viewed immediately by all relevant parties
                (the student who submitted the files, the instructor,
                and the markers).
            \item An interface, with links from which the user can
                download the files he submitted, will be presented
                to him.
        \end{itemize}
    \item Triggers:
        \begin{itemize}
            \item The \emph{Upload} button is clicked by the user.
            \item A file submit interface is presented to the user.
            \item The user selects some files he wishes to submit from his
                local computer and then clicks the \emph{Submit} button.
        \end{itemize}
    \item Basic flow:
        \begin{enumerate}
            \item The files that the user wished to submit are encrypted and
                sent to the server.
            \item The server saves the files to the file system and creates
                database records for those files.
            \item An interface with links to those files is prepared and
                returned.
        \end{enumerate}
\end{itemize}

\item View files:
\begin{itemize}
    \item Primary actor: a registered student.
    \item User story: the user wishes to view a file in his assignment.
    \item Preconditions:
        \begin{itemize}
            \item The user is logged in to the system.
            \item The user has previously submitted some files for an
                assignment.
            \item The user is at an interface where a list of file names
                corresponding to the files he submitted is displayed in a table,
                and each file name is followed by a series of buttons,
                one of which is the \emph{View} button.
        \end{itemize}
    \item Postconditions:
        \begin{itemize}
            \item The file can be displayed (e.g., a plain text or PDF file): 
                an interface with the requested file
                displayed will be presented to the user.
            \item The file cannot be displayed (e.g., a binary file such as an executable
                file): a download dialogue for that file will be presented to
                the user.
        \end{itemize}
    \item Triggers: 
        One of the \emph{View} buttons is clicked.
    \item Basic flow:
        \begin{enumerate}
            \item The request with the file name is encrypted and sent to the server.
            \item The server examines the request and makes sure the user
                has permission to view the requested file.
            \item If the file can be displayed, the system will prepare the
                file (such as syntax highlighting for plain text files) and
                have the interface with that file returned; otherwise, a download
                URL to that file will be returned.
        \end{enumerate}
\end{itemize}
\end{enumerate}


\subsection{Assignment source code control}
The system shall provide abilities for students to keep track of their
assignment source code, that is, keeping a change log for each source code
file so that students can compare the changes every time they submit a new
version of an assignment file.

\subsubsection{Use cases}
\begin{enumerate}
\item View changes:
\begin{itemize}
    \item Primary actor: a registered student.
    \item User story: the user wishes to see the changes between two
        submissions of his assignment.
    \item Preconditions:
        \begin{itemize}
            \item The user is logged in to the system.
            \item The user has made at least two submissions on one source code
                file.
            \item The user is at the file display interface of the file for
                which he
                wishes to view the changes, where there is a \emph{View Changes}
                button on the interface.
        \end{itemize}
    \item Postconditions:
        An interface, with changes on the last two versions of that file
        displayed, is presented to the user.
    \item Triggers:
        The \emph{View Changes} button is clicked.
    \item Basic flow:
        \begin{enumerate}
            \item A \emph{view changes} request with the file identifier is sent
                to the server.
            \item The server examines the request and makes sure the user has
                permission to view the requested file.
            \item If the request is valid, the server will have the current
                version of the source code compared with the previous version,
                then have the result of the comparison formatted and
                returned.
        \end{enumerate}
\end{itemize}
\end{enumerate}

\subsection{Automated code testing}
\label{sec:AUTOTEST}
For programming assignments, the system shall have some routines to build and
run the students' programs automatically with sample data provided by the
instructor, and have the test result available for the users shortly after an
assignment is submitted.

\subsubsection{Use cases}
\begin{enumerate}
\item Configure automated tests for an assignment:
\begin{itemize}
    \item Primary actor: a registered instructor.
    \item User story: the user wishes to enable automated tests for an
        assignment.
    \item Preconditions:
        \begin{itemize}
            \item The user is logged in to the system.
            \item The user has previously created a course.
            \item The user has previously created a programming assignment for
                that course.
            \item The user is at the summary interface of the assignment for
                which he wishes to enable and configure automated tests.
        \end{itemize}
    \item Postconditions:
        \begin{itemize}
            \item An automated test scheme will be created.
            \item Any future submissions will be tested with the scheme.
        \end{itemize}
    \item Triggers:
        \begin{itemize}
            \item The user enables and configures (such as compile commands, 
                time limit, and
                test cases) the automated tests with a dynamic form (a form
                with fields that automatically change according to the previous
                input data) on the
                interface.
            \item The \emph{Save} button at the end of the form is clicked by
                the user.
        \end{itemize}
    \item Basic flow:
        \begin{enumerate}
            \item The form data is validated (basic input validation)
                immediately on the client side of the system (because it is a
                dynamic form). Error messages
                will be displayed if the validation failed.
            \item If the validation is successful, the configuration data is
                serialized into a block of \emph{JSON} \cite{JSON} formatted text and
                encapsulated in a request which then is encrypted and sent to
                the server.
            \item The server examines the request and makes sure the user has
                permission to enable and configure an automated test.
            \item The server decodes the \emph{JSON} text and saves the data to
                the system database.
            \item The server creates a \emph{shell script}~\cite{shellScript}
                for the automated test using the configuration data from the
                system database.
            \item A success ore failure message is returned to the user.
        \end{enumerate}
\end{itemize}

\item View automated test result:
\begin{itemize}
    \item Primary actor: a registered student.
    \item User story: the user wishes to view the automated test result of his
        assignment.
    \item Preconditions:
        \begin{itemize}
            \item The user is logged in to the system.
            \item The user has submitted a programming assignment whose
                automated test had been enabled prior to the submission.
        \end{itemize}
    \item Postconditions:
        An interface with the automated test result will be presented to the
        user.
    \item Triggers: 
        The \emph{Test Result} button is clicked by the user.
    \item Basic flow:
        \begin{enumerate}
            \item A request for the test result is encrypted and sent to the
                server.
            \item The server examines the request and makes sure the user has
                permission to view the requested test result.
            \item The server fetches the test result and performs syntax
                highlighting upon the result.
            \item An interface with the test result is returned.
        \end{enumerate}
\end{itemize}
\end{enumerate}

\subsection{Assignment marking}
The system shall provide tools for instructors and teaching assistants to grade
the students' assignment.

\subsubsection{Use cases}
\begin{enumerate}
\item Grade assignments:
\begin{itemize}
    \item Primary actor: a registered student.
    \item User story: the user wishes to grade an assignment.
    \item Preconditions:
        \begin{itemize}
            \item The user is logged in to the system.
            \item The user is registered as a teaching assistant in the course
                to which the assignment belongs.
            \item The user is at the teaching assistant interface where there
                is a table displaying all the submitted assignments.
        \end{itemize}
    \need 1 in
    \item Postconditions:
        \begin{itemize}
            \item The grade and feedback from the user will be saved and be
                available for the student who submitted the assignment as soon
                as the instructor releases the grades.
        \end{itemize}
    \item Triggers:
        The user clicked the \emph{Mark} button on one of the submitted
        assignments.
    \item Basic flow:
        \begin{enumerate}
            \item A \emph{mark assignment} request is sent to the server.
            \item The server examines the request and makes sure the user has
                marker (see Section~\ref{sec:ASNMAN}) permission.
            \item If the request is valid, a marking interface is returned to
                the user; otherwise, a page with an \emph{unauthorized} error
                message is returned to the user.
            \item A marking interface composed of two sections, 
                a files (that the student submitted) display section and
                a comments section, is presented to the user.
            \item The user checks the files and edits his comments and the
                grade to this assignment until he is satisfied.
            \item The user selects the \emph{Save} button.
            \item The comments and grade are encapsulated in a request which
                then is encrypted and sent to the server.
            \item The server examines the request and makes sure the user has
                permission to grade this assignment.
            \item The server saves the grade and the feedback to the system
                database.
            \item A success message is returned to the user.
        \end{enumerate}
\end{itemize}
\end{enumerate}

\pagebreak

\section{Non-functional requirements}
Following is a list of non-functional requirements ordered by their priorities.

\subsection{Development time}
\label{TIMEFRAME}
The process of development of this project shall be no longer than three
months (starting from May 1, 2018).

\subsection{Security}
\label{SECURITY}
The system shall provide a reasonable degree of security.
That is, reasonable efforts will be made to prevent the release of 
sensitive information (such as student information, grades, and
submitted files) to unauthorized parties utilizing \emph{SQL Injection} 
and/or other forms of cyber attacks \cite{cyberAttacks}.

\subsection{Maintainability}
\label{MAINTAINABILITY}
It is highly possible that future developments will be made to the system by
developers other than the original author, therefore, this system shall be
well designed and constructed with high-quality program code to provide
maintainability.

\subsection{Cross-platform availability}
To provide a better cross-platform availability, this system shall be a
web-based application.

\subsection{Deployment}
The process of deployment of this system shall not be too complicated and
preferably with as few packages required as possible for the host operating
system.