
%-----------------------------------------------------------------------------
% Chapter: Introduction
%-----------------------------------------------------------------------------

\chapter{Requirements Analysis}
\label{chap:REQS}

The first step of this project is performing an overall requirements analysis
in order to determine the general needs of the \emph{Course2018} LMS.
In this chapter, some key requirements will be investigated for their
feasibility and expectations.
Depends on the specifications of the requirements, they are categorized into
two categories, functional requirements and non-functional requirements.

\section{Terminologies}

\subsection{Functional requirement}
A functional requirement defines a task the system needs to accomplish,
as well as the behavior of the task, which is captured and described in detail
with its use cases.
These requirements will be further analyzed to design a system components
model.~\cite{functionalReqs}

\subsection{Non-functional requirement}
As opposed to functional requirements, non-functional requirements describe
non-behavioral specifications such as system performance, security, and system
environments.

\subsection{Use case}
A use case is a series of actions performed by one or more actors interact
with the system to have a task accomplished. \cite{useCase}

\medskip 

Use cases in this thesis are composed with the following elements:

\begin{itemize}

\item \textbf{Primary actor}
The main user (could be a person of an external system) who interacts with the
system.

\item \textbf{User story}
A brief description of the task that the primary actor wishes to accomplish.

\item \textbf{Preconditions}
Prerequisites to the test case;
states or conditions that must be satisfied prior to the start of the workflow.

\item \textbf{Postconditions}
States or conditions that must be satisfied after the workflow is finished.

\item \textbf{Triggers}
An event (or a series of events) that starts the execution of the workflow.

\item \textbf{Basic flow}
A step-by-step description of the detail interactions between the actor and
the system, as well as the activities performed by the system.

\end{itemize}

\section{Functional requirements}

\subsection{User authentication}

The system shall provide routines to handle activities related to user accounts
such as user login, and permission checking for restricted contents.

\subsubsection{Use cases}

\begin{enumerate}
\item User login
\begin{itemize}
\item Primary actor:
    A registered user of the \emph{Course2018} LMS
\item User story:
    A user wishes to have access to some restricted contents, therefore, she
    first need to login to the system.
\item Preconditions:
    User is on the login interface, whose login credential
    (namely, username and password) is input the corresponding fields.
\item Postconditions:
    \begin{itemize}
        \item On success: An interface with the user's information (such as courses
            in which this user is registered) will be presented to the user.
        \item On failure: A login interface with an error message will be
            presented to the user.
    \end{itemize}
\item Triggers: The \emph{Login} button is selected.
\item Basic flow:
    \begin{enumerate}
        \item User's login credential is encrypted and sent to the server.
        \item The server has the user's login credential compared against the
            account database.
        \item If a match is found in the database, an interface with the user's
            information will be returned and the user will have access to
            some restricted contents to which she is authorized; otherwise a
            login interface with error message will be return.
    \end{enumerate}
\end{itemize}

\item Permission checking
\begin{itemize}
\item Primary actor: 
    A registered user of the \emph{Course2018} LMS
\item User story:
    The user request for accessing content to which she may, or may not have
    permission.
\item Preconditions:
    \begin{itemize}
        \item The user has logged-in to the system.
        \item The user has the request url to the content typed into the
            browser.
    \end{itemize}
\item Postconditions:
    \begin{itemize}
        \item On success: An interface with the requested content will be presented
            to the user.
        \item On failure: An interface indicates that the user is unauthorized will
            be presented to the user.
    \end{itemize}
\item Triggers: The user confirms the request.
\item Basic flow:
    \begin{enumerate}
        \item The request (i.e., the url) of the user is sent to the server.
        \item The server checks if the request is valid, namely, if the user
            has logged-in and if the user has permission to the requested
            content.
        \item If the request is valid, an interface with the requested content will
            be returned; otherwise, an interface with an error message indicates
            the request is unauthorized will be returned.
    \end{enumerate}
\end{itemize}

\end{enumerate}


\subsection{Course information handling}
The system shall provide instructors abilities to manage (i.e., create, remove,
and modify) their courses.
\subsubsection{Use cases}
\begin{enumerate}
\item Create courses
\begin{itemize}
    \item Primary actor: A registered instructor
    \item User story: The user wishes to create a course.
    \item Preconditions: 
        \begin{itemize}
            \item The user has logged-in to the system.
            \item The user is on the create course interface where a form with
                course information related fields is presented to the user.
            \item The user has filled in all the fields with data that may, or
                may not be invalid.
        \end{itemize}
    \item Postconditions:
        \begin{itemize}
            \item On success: A summary interface of the course the user created
                will be presented.
            \item On failure: The same create course interface with error
                messages indicating which fields were filled with invalid data
                will be presented.
        \end{itemize}
    \item Triggers: The \emph{Create} button is selected.
    \item Basic flow:
        \begin{enumerate}
            \item The form with the update course information is encrypted and
                sent to the server.
            \item The server validates the form data for errors including,
                but not limited to invalid input format, number out-of-range,
                and conflicts (such as course with the same identifier  
                already exists in that academic year).
            \item If the validation failed, the same create course interface
                with error messages will be return; otherwise the server will
                execute the routines to create a course with the submitted 
                information and have the summary interface of that course prepared
                and returned.
        \end{enumerate}
\end{itemize}

\item Update course information
\begin{itemize}
    \item Primary actor: A registered instructor
    \item User story: The user wishes to update the information of one course
    \item Preconditions:
        \begin{itemize}
            \item The user has logged-in to the system.
            \item The user has previously created a course.
            \item The user is on the manage course interface for the course she
                wishes to modify, where a form with
                the course's existing information pre-filled is presented to
                her.
            \item The user has update some fields with some new data that may,
                or may not be invalid.
        \end{itemize}
    \item Postconditions:
        \begin{itemize}
            \item On success: A manage course interface where a form with
                the course's updated information pre-filled will be presented
                to the user.
            \item On failure: A manage course interface with error messages
                indicating which fields were filled with invalid data will be
                presented to the user.
        \end{itemize}
    \item Triggers: The \emph{Save Changes} button is selected.
    \item Basic flow:
        \begin{enumerate}
            \item The form with course related data is encrypted and sent to
                the server.
            \item The server validates the form data for errors including,
                but not limited to invalid input format, number out-of-range,
                and conflicts (such as course with the same identifier  
                already exists in that academic year).
            \item If the validation failed, the same create course interface
                with error messages will be return; otherwise the server will
                execute the routines to update the course with the submitted 
                information and have the new manage course interface with the
                updated course data prepared and returned.
        \end{enumerate}
\end{itemize}

\item Remove a course
\begin{itemize}
    \item Primary actor: A registered instructor
    \item User story: The user wishes removed a course from the system
    \item Preconditions:
        \begin{enumerate}
            \item The user has logged-in to the system.
            \item The user has previously created a course.
            \item The user is on the manage course interface for the course
                she wishes to remove.
            \item The user selected the \emph{Delete Course} button and a
                confirmation dialogue is presented to her.
        \end{enumerate}
    \item Postconditions:
        The course, along with all relevant data (such as registered students,
        assignments, and student grades) will be removed from the system
        database.
    \item Triggers: The \emph{Confirm} button on the confirmation dialogue is
        selected.
    \item Basic flow:
        \begin{enumerate}
            \item A delete request with the course identifier is encrypted
                and sent to the server.
            \item The server examines the request and make sure the user is
                authorized to perform such an activity.
            \item If the request is valid, the course along with all its
                relevant data will be removed from the system database.
        \end{enumerate}
\end{itemize}

\end{enumerate}

\subsection{Assignment information handling}
The system shall provide abilities for instructors to manage (i.e., create,
remove, and modify) the assignments of their courses, as well as abilities for
students to view and submit those assignments.

\subsubsection{Use cases}
\begin{enumerate}
\item Create assignments
\begin{itemize}
    \item Primary actor: A registered instructor
    \item User story: The user wishes to create an assignment.
    \item Preconditions:
        \begin{itemize}
            \item The user has logged-in to the system.
            \item The user has previously created a course.
            \item The user is on the create assignment interface where a form
                with several basic assignment information related fields is
                presented to the user.
            \item The user has filled in all the fields with data that may,
                or may noy be invalid.
        \end{itemize}
    \item Postconditions:
        \begin{itemize}
            \item On success: An summary interface of the assignment the user
                created will be presented.
            \item On failure: The same create assignment interface with error
                messages indicating which fields were filled with invalid data
                will be presented.
        \end{itemize}
    \item Triggers: The \emph{Create} button is selected.
    \item Basic flow:
        \begin{enumerate}
            \item The form with assignments related data is encrypted and sent
                to the server.
            \item The server validates the form data for errors  including,
                but not limited to invalid input format,  number out-of-range,
                and conflicts (such as assignment with the same number already
                exists in the course).
            \item If the validation failed, the same create assignment interface
                with error messages will be returned; otherwise the server will
                execute the routines to create an assignment with the submitted
                information and have the summary interface of that assignment
                prepared and returned.
        \end{enumerate}
\end{itemize}

\item Update assignment information
\begin{itemize}
    \item Primary actor: A registered instructor
    \item User story: The user wishes to update the information of one
        assignment.
    \item Preconditions:
        \begin{itemize}
            \item The user has logged-in to the system.
            \item The user has previously created a course.
            \item The user has previously created an assignment for that course.
            \item the user is on the summary interface of the assignment she
                wishes to modify, where a form with the assignment's existing
                information pre-filled is presented to her.
        \end{itemize}
    \item Postconditions:
        \item On success: The same assignment summary interface with the updated
            information will be presented to the user.
        \item On failure: The same assignment summary interface with error
            messages indicating which fields were filled with invalid data will
            be presented to the user.
    \item Triggers: The \emph{Save Changes} button is selected.
    \item Basic flow:
        \begin{enumerate}
            \item The form with the updated assignment information is encrypted
                and sent to the server.
            \item The server validates the form data for the same errors
                mentioned in the previous use case.
            \item If the validation failed, the same create assignment interface
                with error messages will be returned; otherwise the server will
                execute the routines to update the assignment with the submitted
                information and have the new summary interface with the updated
                information prepared and returned.
        \end{enumerate}
\end{itemize}

\item Remove an assignment
\begin{itemize}
    \item Primary actor: A registered instructor
    \item User story: The user wishes to have one assignment removed.
    \item Preconditions:
        \begin{itemize}
            \item The user has logged-in to the system.
            \item The user has previously created a course.
            \item The user has previously created an assignment for that course.
            \item the user is on the summary interface of the assignment she
                wishes to remove.
            \item The user selected the \emph{Delete Assignment} button and a
                confirmation dialogue is presented to her.
        \end{itemize}
    \item Postconditions:
        The assignment, along with all the relevant data (such as assignment
        attachments, student grades and marker comments) will be removed from
        the system database.
    \item Triggers: The \emph{Confirm} button on the confirmation dialogue is
        selected.
    \item Basic flow:
        \begin{enumerate}
            \item A delete request with the assignment identifier is encrypted
                and sent to the server.
            \item The server examines the request and make sure the user is
                authorized to perform such an activity.
            \item If the request is valid, the assignment along with all its
                relevant data will be removed from the system database.
        \end{enumerate}
\end{itemize}

\item View assignment
\begin{itemize}
    \item Primary actor: A registered student
    \item User story: The user wishes to view the information of an assignment
     (such as assignment descriptions and attachments) .
    \item Preconditions:
        \begin{itemize}
            \item The user has logged-in to the system.
            \item The user is registered in the course whose instructor has
                prepared some assignments.
        \end{itemize}
    \item Postconditions: An interface with the assignment information the user
        wishes to view will be presented.
    \item Triggers: The user selected a link to the assignment detail interface.
    \item Basic flow:
        \begin{enumerate}
            \item A request with the assignment identifier is encrypted and
                sent to the server.
            \item The server examines the request and make sure the user has
                permission to access the requested content (i.e., the user is
                registered in the course to which the assignment belongs).
            \item If the request is valid, an interface with the requested
                content will be prepared and return.
        \end{enumerate}
\end{itemize}
\end{enumerate}

\subsection{File handling}
The system shall provide abilities for users to upload, view, and download
files when they need to (such as submit assignments and upload attachments).

\subsubsection{Use cases}
\begin{enumerate}

\item Upload files
\begin{itemize}
    \item Primary actor: A registered student
    \item User story: The user wishes to submit an assignment.
    \item Preconditions:
        \begin{itemize}
            \item The user has logged-in to the system.
            \item The user is registered in the course for which he wishes to
                submit an assignment.
            \item The user is on the submit assignment interface where an
                \emph{Upload} button is presented to him.
        \end{itemize}
    \item Postconditions:
        \begin{itemize}
            \item The files the user submitted will be saved to the system.
            \item Theses files can be view immediately by all relevant parties
                (the student submitted the files, the instructor,
                and the markers).
            \item An interface, with links from which the user can
                download the files he submitted, will be presented
                to him.
        \end{itemize}
    \item Triggers:
        \begin{itemize}
            \item The \emph{Upload} button is selected by the user.
            \item A file submit interface is presented to the user.
            \item The user selected some files he wishes to submit from his
                local hard drive and then selected the \emph{Submit} button.
        \end{itemize}
    \item Basic flow:
        \begin{enumerate}
            \item The files that the user wished to submit are encrypted and
                sent to the server.
            \item The server save the files to file system and creates database
                records for those files.
            \item An interface with links to those files will be prepared and
                returned.
        \end{enumerate}
\end{itemize}

\item View files
\begin{itemize}
    \item Primary actor: A registered student
    \item User story: The user wishes to view a file in his assignment
    \item Preconditions:
        \begin{itemize}
            \item The user has logged-in to the system.
            \item The user has previously submitted some files for an
                assignment.
            \item The user is on an interface where a list of file names
                corresponding to the files he submitted is displayed on a table,
                each file name is followed by a series of buttons, one of which
                is the \emph{View} button.
        \end{itemize}
    \item Postconditions:
        \begin{itemize}
            \item File can be displayed (such as plain text or pdf file): 
                An interface with the requested file
                displayed will be presented to the user.
            \item File cannot be displayed (binary file such as an executable
                file): A download dialogue for that file will be presented to
                the user.
        \end{itemize}
    \item Triggers: 
        One of \emph{View} button is selected
    \item Basic flow:
        \begin{enumerate}
            \item The request with the file name is sent to the server.
            \item The server examines the request, and make sure the user
                has permission to view the requested file.
            \item If the file can be displayed, the system will prepare the
                file (such as syntax highlighting for plain text file) and
                have the interface with that file returned; otherwise a download
                url to that file will be returned.
        \end{enumerate}
\end{itemize}
\end{enumerate}


\subsection{Assignment source code control}
The system shall provide abilities for students to keep track of their
assignment source code, that is, keeping a changing log to each source code
file so that students can compare the changes every time they submit an
assignment.

\subsubsection{Use cases}
\begin{enumerate}
\item View changes
\begin{itemize}
    \item Primary actor: A registered student
    \item User story: The user wishes to see the changes between two
        submissions of his assignment.
    \item Preconditions:
        \begin{itemize}
            \item The user has made at least two submissions on one source code
                file.
            \item The user is on the file display interface of the file he
                wishes to view the changes, where there is a \emph{View Changes}
                button on the interface.
        \end{itemize}
    \item Postconditions:
        An interface with changes on that file displayed is presented to the
        user.
    \item Triggers:
        The \emph{View Changes} button is selected.
    \item Basic flow:
        \begin{enumerate}
            \item A view changes request with the file identifier is sent
                to the server.
            \item The server examines the request and make sure the user has
                permission to view the requested file.
            \item If the request is valid, the server will have the current
                version of the source code compared with the previous version,
                then have the result of the comparison formatted and
                returned.
        \end{enumerate}
\end{itemize}
\end{enumerate}

\subsection{Automatic code testing}
\subsubsection{Use cases}
\begin{enumerate}
\item 
\begin{itemize}
    \item Primary actor:
    \item User story:
    \item Preconditions:
    \item Postconditions:
    \item Triggers:
    \item Basic flow:
        \begin{enumerate}
            \item 
        \end{enumerate}
\end{itemize}
\end{enumerate}

\subsection{Assignment Marking}
\subsubsection{Use cases}
\begin{enumerate}
\item 
\begin{itemize}
    \item Primary actor:
    \item User story:
    \item Preconditions:
    \item Postconditions:
    \item Triggers:
    \item Basic flow:
        \begin{enumerate}
            \item 
        \end{enumerate}
\end{itemize}
\end{enumerate}


\section{Non-functional requirements}
\subsection{Security}
\subsection{Maintainability}
\subsection{Cross platform availability}
\subsection{Deployment}