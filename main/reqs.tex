
%-----------------------------------------------------------------------------
% Chapter: Introduction
%-----------------------------------------------------------------------------

\chapter{Requirements Analysis}
\label{chap:REQS}

The first step of this project is performing an overall requirements analysis
in order to determine the general needs of the \emph{Course2018} LMS.
In this chapter, some key requirements will be investigated for their
feasibility and expectations.
Depends on the specifications of the requirements, they are categorized into
two categories, functional requirements and non-functional requirements.

\section{Terminologies}

\subsection{Functional requirement}
A functional requirement defines a task the system needs to accomplish,
as well as the behavior of the task, which is captured and described in detail
with its use cases.
These requirements will be further analyzed to design a system components
model.~\cite{functionalReqs}

\subsection{Non-functional requirement}
As opposed to functional requirements, non-functional requirements describe
non-behavioral specifications such as system performance, security, and system
environments.

\subsection{Use case}
A use case is a series of actions performed by one or more actors interact
with the system to have a task accomplished. \cite{useCase}

\medskip 

Use cases in this thesis are composed with the following elements:

\begin{itemize}

\item \textbf{Primary actor}
The main user (could be a person of an external system) who interacts with the
system.

\item \textbf{User story}
A brief description of the task that the primary actor wishes to accomplish.

\item \textbf{Preconditions}
Prerequisites to the test case;
states or conditions that must be satisfied prior to the start of the workflow.

\item \textbf{Postconditions}
States or conditions that must be satisfied after the workflow is finished.

\item \textbf{Triggers}
An event (or a series of events) that starts the execution of the workflow.

\item \textbf{Basic flow}
A step-by-step description of the detail interactions between the actor and
the system, as well as the activities performed by the system.

\end{itemize}

\section{Functional requirements}

\subsection{User authentication}

The system shall provide routines to handle activities related to user accounts
such as user login, and permission checking for restricted contents.

\subsubsection{Use cases}

\begin{enumerate}
\item User login
\begin{itemize}
\item Primary actor:
    A registered user of the \emph{Course2018} LMS
\item User story:
    A user wishes to have access to some restricted contents, therefore, she
    first need to login to the system.
\item Preconditions:
    User is on the login interface, whose login credential
    (namely, username and password) is input the corresponding fields.
\item Postconditions:
    \begin{itemize}
        \item On success: An interface with the user's information (such as courses
            in which this user is registered) will be presented to the user.
        \item On failure: A login interface with an error message will be
            presented to the user.
    \end{itemize}
\item Triggers: The \emph{Login} button is selected.
\item Basic flow:
    \begin{enumerate}
        \item User's login credential is encrypted and sent to the server.
        \item The server has the user's login credential compared against the
            account database.
        \item If a match is found in the database, an interface with the user's
            information will be returned and the user will have access to
            some restricted contents to which she is authorized; otherwise a
            login interface with error message will be return.
    \end{enumerate}
\end{itemize}

\item Permission checking
\begin{itemize}
\item Primary actor: 
    A registered user of the \emph{Course2018} LMS
\item User story:
    The user request for content to which she may, or may not have permission
    to access.
\item Preconditions:
    \begin{itemize}
        \item The user has logged-in to the system.
        \item The user has the request url to the content typed into the
            browser.
    \end{itemize}
\item Postconditions:
    \begin{itemize}
        \item On success: An interface with the requested content will be presented
            to the user.
        \item On failure: An interface indicates that the user is unauthorized will
            be presented to the user.
    \end{itemize}
\item Triggers: The user confirms the request.
\item Basic flow:
    \begin{enumerate}
        \item The request (i.e., the url) of the user is sent to the server.
        \item The server checks if the request is valid, namely, if the user
            has logged-in and if the user has permission to the requested
            content.
        \item If the request is valid, an interface with the requested content will
            be returned; otherwise, an interface with an error message indicates
            the request is unauthorized will be returned.
    \end{enumerate}
\end{itemize}

\end{enumerate}


\subsection{Course information handling}
The system shall provide instructors abilities to manage (i.e., create, remove,
and modify) their courses.
\subsubsection{Use cases}
\begin{enumerate}
\item Create courses
\begin{itemize}
    \item Primary actor: A registered instructor
    \item User story: The user wishes to create a course.
    \item Preconditions: 
        \begin{itemize}
            \item The user has logged-in to the system.
            \item The user is on the create course interface where a form with
                course information related fields is presented to the user.
            \item The user has filled in all the fields with data that may, or
                may not be invalid.
        \end{itemize}
    \item Postconditions:
        \begin{itemize}
            \item On success: A summary interface of the course the user created
                will be presented.
            \item On failure: The same create course interface with error
                messages indicating which fields were filled with invalid data
                will be presented.
        \end{itemize}
    \item Triggers: The \emph{Create} button is selected.
    \item Basic flow:
        \begin{enumerate}
            \item The form with course related data is encrypted and sent to
                the server.
            \item The server validates the form data for errors including,
                but not limited to invalid input format, number out-of-range,
                and conflicts (such as course with the same identifier  
                already exists in that academic year).
            \item If the validation failed, the same create course interface
                with error messages will be return; otherwise the server will
                execute the routines to create a course with the submitted 
                information and have the summary interface of that course prepared
                and returned.
        \end{enumerate}
\end{itemize}

\item Update course information
\begin{itemize}
    \item Primary actor: A registered instructor
    \item User story: The user wishes to update the information of one course
    \item Preconditions:
        \begin{itemize}
            \item The user has logged-in to the system.
            \item The user has previously created a course.
            \item The user is on the manage course interface for the course she
                wishes to modify, where a form with
                the course's existing information pre-filled is presented to
                the her.
            \item The user has update some fields with some new data that may,
                or may not be invalid.
        \end{itemize}
    \item Postconditions:
        \begin{itemize}
            \item On success: A manage course interface where a form with
                the course's updated information pre-filled will be presented
                to the user.
            \item On failure: A manage course interface with error messages
                indicating which fields were filled with invalid data will be
                presented to the user.
        \end{itemize}
    \item Triggers: The \emph{Save Changes} button is selected.
    \item Basic flow:
        \begin{enumerate}
            \item The form with course related data is encrypted and sent to
                the server.
            \item The server validates the form data for errors including,
                but not limited to invalid input format, number out-of-range,
                and conflicts (such as course with the same identifier  
                already exists in that academic year).
            \item If the validation failed, the same create course interface
                with error messages will be return; otherwise the server will
                execute the routines to update the course with the submitted 
                information and have the new manage course interface with the
                updated course data prepared and returned.
        \end{enumerate}
\end{itemize}

\item Remove a course
\begin{itemize}
    \item Primary actor: A registered instructor
    \item User story: The user wishes removed a course from the system
    \item Preconditions:
        \begin{enumerate}
            \item The user has logged-in to the system.
            \item The user has previously created a course.
            \item The user is on the manage course interface for the course
                she wishes to remove.
            \item The user click the \emph{Delete Course} button and a
                confirmation dialogue is presented to her.
        \end{enumerate}
    \item Postconditions:
        The course, along with all relevant data (such as registered students,
        assignments, and student grades) will be removed from the system
        database.
    \item Triggers: The \emph{Confirm} button is selected.
    \item Basic flow:
        \begin{enumerate}
            \item A request with the course identifier will be encrypted and
                sent to the server.
            \item The server examines the request and make sure the user is
                authorized to perform such an activity.
            \item If the request is valid, the course along with all its
                relevant data will be removed from the system database.
        \end{enumerate}
\end{itemize}

\end{enumerate}

\subsection{Assignment information handling}
The system shall provide abilities for instructors to manage (i.e., create,
remove, and modify) the assignments of their courses, as well as abilities for
students to view and submit those assignments.

\subsubsection{Use cases}
\begin{enumerate}
\item 
\begin{itemize}
    \item Primary actor:
    \item User story:
    \item Preconditions:
    \item Postconditions:
    \item Triggers:
    \item Basic flow:
        \begin{enumerate}
            \item 
        \end{enumerate}
\end{itemize}
\end{enumerate}

\subsection{Assignment source code control}
\subsubsection{Use cases}
\begin{enumerate}
\item 
\begin{itemize}
    \item Primary actor:
    \item User story:
    \item Preconditions:
    \item Postconditions:
    \item Triggers:
    \item Basic flow:
        \begin{enumerate}
            \item 
        \end{enumerate}
\end{itemize}
\end{enumerate}

\subsection{Automatic code testing}
\subsubsection{Use cases}
\begin{enumerate}
\item 
\begin{itemize}
    \item Primary actor:
    \item User story:
    \item Preconditions:
    \item Postconditions:
    \item Triggers:
    \item Basic flow:
        \begin{enumerate}
            \item 
        \end{enumerate}
\end{itemize}
\end{enumerate}

\subsection{File handling}
\subsubsection{Use cases}
\begin{enumerate}
\item 
\begin{itemize}
    \item Primary actor:
    \item User story:
    \item Preconditions:
    \item Postconditions:
    \item Triggers:
    \item Basic flow:
        \begin{enumerate}
            \item 
        \end{enumerate}
\end{itemize}
\end{enumerate}

\subsection{Assignment Marking}
\subsubsection{Use cases}
\begin{enumerate}
\item 
\begin{itemize}
    \item Primary actor:
    \item User story:
    \item Preconditions:
    \item Postconditions:
    \item Triggers:
    \item Basic flow:
        \begin{enumerate}
            \item 
        \end{enumerate}
\end{itemize}
\end{enumerate}


\section{Non-functional requirements}
\subsection{Security}
\subsection{Maintainability}
\subsection{Cross platform availability}
\subsection{Deployment}