
%-----------------------------------------------------------------------------
% Chapter: Introduction
%-----------------------------------------------------------------------------

\chapter{Requirements Analysis}
\label{chap:REQS}

The first step of this project is performing an overall requirements analysis
in order to determine the general needs of the \emph{Course2018} LMS.
In this chapter, some key requirements will be investigated for their
feasibility and expectations.
Depends on the specifications of the requirements, they are categorized into
two categories, functional requirements and non-functional requirements.

\section{Terminologies}

\subsection{Functional requirement}
A functional requirement defines a task the system needs to accomplish,
as well as the behavior of the task, which is captured and described in detail
with its use cases.
These requirements will be further analyzed to design a system components
model.~\cite{functionalReqs}

\subsection{Non-functional requirement}
As opposed to functional requirements, non-functional requirements describe
non-behavioral specifications such as system performance, security, and system
environments.

\subsection{Use case}
A use case is a series of actions performed by one or more actors interact
with the system to have a task accomplished. \cite{useCase}

\medskip 

Use cases in this thesis are composed with the following elements:

\begin{itemize}

\item \textbf{Title}
A brief summary of the use case.

\item \textbf{Primary actor}
The main user (could be a person of an external system) who interacts with the
system.

\item \textbf{User story}
Natural language description of features of the system involved in the test
case that are essential to the task be accomplished.

\item \textbf{Preconditions}
Prerequisites to the test case.
States or conditions that must be satisfied prior to the start of the workflow.

\item \textbf{Postconditions}
States or conditions that must be satisfied after the workflow is finished.

\item \textbf{Triggers}
An event (or a series of events) that starts the execution of the workflow

\item \textbf{Basic flow}
A step-by-step description of the detail interactions between the actor and
the system, as well as the activities performed by the system.

\end{itemize}



\section{Functional requirements}

\subsection{User authentication}
\subsubsection{Definition}
\subsubsection{Use cases}

% \subsection{Course information handling}
% \subsubsection{Definition}
% \subsubsection{Use cases}

% \subsection{Assignment information handling}
% \subsubsection{Definition}
% \subsubsection{Use cases}

% \subsection{Automatic code testing}
% \subsubsection{Definition}
% \subsubsection{Use cases}

% \subsection{File handling}
% \subsubsection{Definition}
% \subsubsection{Use cases}

% \subsection{Assignment Marking}
% \subsubsection{Definition}
% \subsubsection{Use cases}


% \section{Non-functional requirements}
% \subsection{Security}
% \subsection{Maintainability}
% \subsection{Cross platform availability}
% \subsection{Deployment}