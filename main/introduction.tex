
%-----------------------------------------------------------------------------
% Chapter: Introduction
%-----------------------------------------------------------------------------

\chapter{Introduction}
\label{chap:INTRO}

Many courses in computer science involve programming in one or more programming
languages.
The nature of programming assignment evaluation
involves processes like code testing, debugging, commenting, and code style
checking.
This makes the evaluation process of a programming computer science assignment
very different than assignments in many other disciplines.
Students would be motivated to hand in a higher quality assignment
if more information were provided shortly after their assignments were
submitted.
A learning management system (LMS) that can automatically test the
submitted assignments and have the results provided for both the markers and
the students is desirable.

\section{Background}

Acadia University currently uses the Moodle Learning System (also known at 
Acadia University as Acorn) as the default LMS.
This system works well for most of the courses, however, it has shortcomings
when it comes to computer science programming assignments.
Due to the nature of programming source code files,
the Moodle LMS lacks certain features that are essential to computer science
students and instructors,
including, but not limited to, monospace text font for plain text
file display, programming language detection, and syntax
highlighting for source code file display.
Moreover, it is not uncommon for computer science markers and instructors
to want to put mathematical equations or formulas into their comments on
students' assignments; the only ways to do that with the Moodle
LMS are to draw the equations or formulas by hand with the
pointing device of the computer (i.e., computer mouse, trackpad), write
the them in \emph{HTML} code,
or write them in another software then and insert to the Moodle LMS as an image.
Both of these ways are
time-consuming and borderline unfeasible.

\medskip

The \emph{Course2000} course system, designed by former Acadia computer science
professor Dr.~Rick Giles, is a good choice
of alternative LMS to the Moodle Learning System.
As a matter of fact, it has been the
default LMS for one computer science course, \emph{COMP 2103: Programming 3},
for over 16 years.
With a recent addition to the system,
the \emph{Visual Mark} module,
designed and integrated by former student Tim Cooper,
the \emph{Course2000} system has proven itself
to be very capable of handling computer science assignments.
However, with more
and more modifications applied to the system over the years, maintaining
this system has become more and more difficult.
Further, the amount of work required to add new features to the
system (especially the key features this thesis intends to achieve,
see Section~\ref{FEATURES}) has increased dramatically
due to its legacy codebase.
For example, the whole system frontend of the \emph{Course2000} LMS was built
with an \emph{HTML} feature called \emph{framing}, which is an obsolete feature
and it had been dropped in \emph{HTML5}~\cite{framing}, 
as a result, a lot of changes to the \emph{Course2018} would require a complete
rewrite of the whole system frontend.

\section{Course2018}
Influenced by the \emph{Course2000} system, the goal of this project is to
develop a secure and dynamic LMS with functionalities designed specifically for
computer science instructors and students.
These functionalities are mainly focused on
providing informative feedback on programming assignments to both students
and instructors for the purpose of easing the process of learning as well as
evaluation.
Simplicity of the overall system design is an important goal of this project.
Every task the system 
performs can be done efficiently for both the users and the server,
and the future development as well as maintenance will not be too difficult.

\subsection{Key features}
\label{FEATURES}
The \emph{Course2018} LMS contains the following functionalities.

\subsubsection{Source code management}
The \emph{Course2018} LMS provides the ability for students to manage
their source code more efficiently and more professionally. With features such
as source code version control, students and instructors are now able
to view the history of every source code file, and more importantly, compare
the changes between different versions of each file.

\subsubsection{Souce code display}
Source code display is one fundamental feature of the \emph{Course2018} LMS,
which provides better readability of source code files for the users.
A syntax highlighter is included in the system for over 300 programming
languages~\cite{ApygmentsLangs}.

\subsubsection{Automated tests for programming assignments}
When a programming assignment is submitted by a student with \emph{Course2018},
a program associated with the submitted source code file(s) will be built
(if required).
This program
will then be executed with the sample input data provided to the system
by the instructor.
As soon as the program execution is finished, 
the output of the program will be compared with the sample output data~(also
provided by the instructor) and
an informative test result will be generated for the relevant parties~(i.e.,
the student who submitted the assignment, the markers, and the
instructor).

\subsubsection{Informative assignment feedback}
The key to providing informative assignment feedback comments in computer
science assignments (especially programming assignments) is to make sure the
comments are well formatted.
Therefore, an easy-to-use plain text markup language called \emph{Markdown}
is introduced to the \emph{Course2018} marking system. This markup language was
designed to be automatically converted to \emph{HTML} format.
Also, \emph{Markdown} is
widely used by well-known websites including \emph{GitHub} \cite{CgitHubMarkDown}
and \emph{Stack Overflow} \cite{stackOverflowMarkdown}.
An improvement has been made to support mixed
input of \emph{Markdown} and \TeX\ mathematical notation.

\subsubsection{Powerful built-in text editor}
With the enhanced assignment feedback solution mentioned above, a powerful
text editor that supports syntax highlighting, and custom key-bindings is also
integrated to the \emph{Course2018} LMS in order to ease the actual process of
typing for the markers and instructors.