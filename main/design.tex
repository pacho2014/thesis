
%-----------------------------------------------------------------------------
% Chapter: System Design
%-----------------------------------------------------------------------------

\chapter{System Design}
\label{chap:SYSDES}

With all the key requirements being laid out in the previous chapter, it is
time to analyze them and have a system design model in an abstract
level drawn. 

\section{Specifications}
Non-functional requirements will be investigated first to decide the
non-behavioral specifications of the system, specifications like the
programming language and framework that powers the system, choice of database,
and environment in which the system will be running.

\subsection{Framework}
Debate on whether or not it is better to use a framework is never failed to be
a popular topic in the developer communities. On one hand, many developers
believe that a framework puts a lot of constrains to a project which limits the
overall control to the project (especially core behaviors) \cite{frameworks};
on the other hand, a framework offers a better degree of efficiency and better
measurements of security to a project, while keeping the simplicity and
the architectural clarity of the project by enforcing the developers to apply
an architectural pattern (such as the \emph{model-view-controller} pattern
\cite{mvc}) to their code.

\medskip

Given the development time frame, requirements for
security and maintainability (see 2.3.1, 2.3.2 and 2.3.3),
as well as the scale of the this project,
it is not unjustifiable for this project to use a web framework to achieve a
better result.

\medskip

The framework this project uses is called \emph{Django}, an open-source web
framework written in \emph{Python}. \emph{Django} was originally developed by
the web programming team at the \emph{Lawrence Journal-World} newspaper in 2003,
dedicated for the the company to deliver their contents as fast as possible
(James Bennett, 2008 \cite{django}). With more than a decade of improvements
being made to the framework, it is now among one of the most popular web
frameworks in the world, used by large companies and organizations like the
\emph{Washington Post} (Adrian Holovaty, 2005 \cite{djangoWashingtonPost})
and \emph{Instagram} (Instagram Engineering, 2011 \cite{djangoInstagram}).
\medskip

The \emph{Django} framework provides a set of libraries to handle most basic
aspects of web application development. One good example of those libraries is
the \emph{object-relational mapper}, the framework has the capability to create
relational database schemas based on the \emph{Python} data model classes
written by the developer, so that no raw SQL query is required
(James Bennett, 2008 \cite{django}) for most of the
database operations (such as select, filter, remove, and insert database
records),
making the possibility of having \emph{SQL Injection} vulnerabilities next
to zero.

\subsection{Database}
The \emph{Course2018} LMS uses \emph{PostgreSQL} as the default database
management system, primarily because its extendibility (such as direct
support of inheritance and document data types), which can be used for
development in the future of more advanced features without worrying the
limitations of the database management system~\cite{postgres}.

\medskip

Furthermore, if future developments really require a change of database
management system,
with the \emph{Django} framework's database libraries,
switching between different database management systems only requires
modification of a few lines of code in the setting file.

\subsection{Environment}
