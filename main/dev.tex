
%-----------------------------------------------------------------------------
% Chapter: Development
%-----------------------------------------------------------------------------

\chapter{Development}
\label{chap:DEV}

\section{Methodology}
The most important point in the development process of the \emph{Course2018}
LMS is the idea of \emph{continuous integration} (CI), meaning that every new
develop feature is tested and integrate to the existing server seamlessly.

\medskip
This is achieved by using a development tool called \emph{GitLab CI}
\cite{gitlabCI}, a sub-system that came with the source management tool,
\emph{GitLab}~\cite{gitlab}, which is also used for the development of this
project. 

\medskip

This toolchain offers abilities to a developer so that every time
the developer finishes programming of a feature and pushes the code to the
\emph{GitLab} project repository, the \emph{GitLab} server then invokes some
routines to test the new code with a test scheme also defined by the developer
prior to the time when the code was pushed; finally, if all tests are passed, 
the \emph{GitLab} server deploys the new version of the project by simply
building a new container that has the latest code in it, and has the old
container replaced with the new one.

\section{Deployment}
In the development process of this project, the deployment stage is defined
before the actual programming even start. This approach makes sure that all
modules of the project can be test in the
deployment environment (i.e., the container in which the \emph{Course2018}
will be eventually running) as soon as the implementation if finished.

\subsection{Environment}
The \emph{Course2018} LMS is containerized in an environment
(the production container) based on
\emph{Debian 8}~\cite{debian}, with \emph{Python 3} and all necessary packages
(including \emph{Django 2.0}) installed.

\subsection{HTTP server}
An \emph{HTTP server} is included in the production container to host the
\emph{Course2018} LMS.
The \emph{HTTP server} in use in the production container is the
\emph{Apache HTTP server} (version 2.4)~\cite{apache}, with the
\texttt{mod\_wsgi}~\cite{wsgi} package installed to accommodate
\emph{Python} web applications in \emph{WSGI}
(Web Server Gateway Interface~\cite{wsgi}) specification, such as a
\emph{Django} project.