
%-----------------------------------------------------------------------------
% Chapter:Future work 
%-----------------------------------------------------------------------------

\chapter{Future Work and Conclusion}
\label{chap:FUTURE}

\section{Future work}
The \emph{Course2018} LMS has a collection of functionalities 
with features specifically designed for computer science programming
classes.
During the implementation of the code and subsequent testing,
various useful additions to the system
described in the previous chapters were identified.
These additions are described in this chapter.

\subsection{User registration interface}
As of now, the \emph{Course2018} LMS uses the \emph{Django Admin} site to
add new users to the system, but this action can only performed by users that
have \texttt{admin} privilege to the system~\cite{AdjangoAdmin}.
Further, this
makes it very inconvenient to import a large amount of user information to
the system, because all the user information has to be input to the system one
user at a time. Hence, a user registration interface is desired.

\subsubsection{Implementation suggestions}
The user registration should be open only to students
who are registered in courses which are using the \emph{Course2018} LMS\null.
An instructor would first upload a file with the students' basic information
(e.g., student ID, first name, last name, and email) to the LMS\null. The LMS would
create user accounts and profiles for those students, and marks all of them
as inactive users (see attribute \texttt{is\_active} in
Table~\ref{tab:USR_ATTR}).
Then a student can input and submit his student ID to the registration
interface; after the LMS receives the student's ID,
it sends an email with an expirable temporary password to the student's
email account.
Finally, the student uses that temporary password to reset his password and
activate his account.

\subsection{Course materials}
Course material (e.g., lecture notes) management is a desired feature to be
added to the \emph{Course2018} LMS\null.

\subsubsection{Implementation suggestions}
The implementation of this feature should be very simple.
As shown in
Figure~\ref{fig:COURSE_SUMMARY}, the \emph{course summary} template is already
designed to be connected to the course materials management module.
Moreover, the \emph{data model} design of this module should
be very similar to the design of the \emph{AssignmentAttachment} model 
(see Table~\ref{tab:ATTACH_FILE_ATTR}).

\subsection{More source code management functionalities}
Since \emph{Git} and \emph{GitLab} are already integrated to the
\emph{Course2018} LMS, the features listed below can be added to the LMS in the
future with a modest amount of effort:
\begin{itemize}
    \item Revert: allows a user to revert one or more of his submitted files to
        previous versions.
    \item Enhanced version comparison: allows a user to compare the differences
        between any two versions of a file, not just the two most recent
        versions.
    \item Issues: allows a user to post issues (e.g., questions about an
        assignment, or objections to some deductions) on assignments, and
        discuss the issues with the instructor or the TAs.
\end{itemize}

\subsection{Automated test configurations for TAs}
Currently, only instructors are allowed to update automated test
configurations of an assignment in the \emph{Course2018} LMS\null.
It is not uncommon for TAs to come up with new test
cases during the process of grading assignments, and it is desireable for TAs
to have the ability to update the automated test configurations of the
assignment problems they are assigned to grade.

\subsection{\emph{REST} APIs}
Representational State Transfer, also known as \emph{REST}, is a standard of
web applications API architectural style commonly used for communications
between clients and the server~\citep[Chapter 5]{REST}.

To implement all features of the \emph{Course2018} LMS to a set of \emph{REST} APIs
may require a considerable amount of work. However, once this implementation is
finished, the LMS can offer abilities for advanced users to interact with the
LMS in any form they
wish. For example, an instructor can implement a script using those APIs to
interact with the LMS and fully automate the process of assignment management
to satisfy his own workflow. In addition, those sets of APIs can be used
to implement a mobile device version of the \emph{Course2018} application.

\subsection{Interact with the \emph{Moodle} LMS}
A lot of instructors at Acadia University are heavily dependent on the
\emph{Moodle} LMS, for some features that are not provided by
\emph{Course2018} (e.g., grade book, quiz/test tools, and midterm/final
exam grade recording). The efficiency of learning management would be
highly increased if the \emph{Course2018} LMS can interact with the existing
\emph{Moodle} LMS\null.


\section{Conclusion}
In conclusion, this thesis work presents a comprehensive
learning management system which mainly focuses on computer science
programming courses. The implementation of this project demonstrates
a rapid application development process with modern development tools
such as \emph{Docker}, \emph{GitLab}, and \emph{GitLab CI}\null.
The final product, \emph{Course2018} LMS, has the potential to expose
students to a software development environment where features like source
code management and
automated tests play significant roles in their assignments, encouraging
students to submit higher quality work.
Further,
the enhanced marking interface allows professors and TAs to provide more
informative feedback to students in a more efficient way.