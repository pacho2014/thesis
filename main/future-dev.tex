
%-----------------------------------------------------------------------------
% Chapter:Future enhancements 
%-----------------------------------------------------------------------------

\chapter{Future enhancements}
\label{chap:FUTURE}

As a part of this thesis work, 
the \emph{Course2018} LMS has all the functionalities that define itself
to be a special and computer-science-facing LMS.
However the lack of certain features also makes it less than ideal in terms
of user experience.
Therefore, some suggestions of future enhancements are pointed out in this
chapter.

\section{User registration interface}
As of now, the \emph{Course2018} LMS uses the \emph{Django Admin} site to
add new users to the system, but this action can only performed by users that
has \texttt{admin} privilege to the system~\cite{djangoAdmin} and it is very
inconvenient to import a large amount of user information to the system.
Hence, a user registration interface is desired.

\subsection{Implementation suggestions}
As a departmental LMS, the user registration should be opened only to students
of the CS department.
That is, an instructor first upload a file with the students' basic information
(e.g., student ID, first name, last name, and email) to the LMS, the LMS
creates user accounts and profiles for those students, and marks all of them
as inactive users (see attribute \texttt{is\_active} in
Table~\ref{tab:USR_ATTR}).
Then a student can input and submit his student ID to the registration
interface; after the LMS receive the student's ID,
it sends an email with an expirable temporary password to the student's
email account.
Finally, the student uses that temporary password to reset his password and
activate his account.

\section{Course materials}
Course materials (e.g., lecture notes) management is a desired feature to be
added to the \emph{Course2018} LMS.

\subsection{Implementation suggestions}
The implementation of this feature should be very simple.
As it is shown in
Figure~\ref{fig:COURSE_SUMMARY}, the \emph{course summary} template is already
designed to be connected to course materials management module.
Moreover, the \emph{data model} design of this module should
be very similar the the design of the \emph{AssignmentAttachment} model 
(see Table~\ref{tab:ATTACH_FILE_ATTR}).

\section{More source code management functionalities}
Since \emph{Git} and \emph{GitLab} is already integrated to the
\emph{Course2018} LMS, features listed below can be added to the LMS in the
future:
\begin{itemize}
    \item Revert: allows a user to revert some of his submitted file to a
        previous version.
    \item Enhanced version comparison: allows a user to compare the differences
        between any two versions of a file.
    \item Issues: allows a user to post issues (e.g., confusions of an
        assignment, or objections of some deductions) on assignments, and
        discusses the issues with the instructor or the TAs.
\end{itemize}

\section{Automated test configurations for TAs}
Right now it is only allowed for instructors to update automated test
configurations of an assignment in the \emph{Course2018} LMS.
One can foresee that it is not uncommon for TAs to come up with new test
cases during the process of grading assignments, and it is desireable for TAs
to have abilities to update the automated test configurations of the assignment
problems he is assigned to grade.

\section{\emph{REST} APIs}
Representational State Transfer, also known as \emph{REST}, is a standard of
web applications API architectural style commonly used for communications
between clients and the server~\citep[Chapter 5]{REST}.

To implement all features of the \emph{Course2018} to a set of \emph{REST} APIs
may require a huge amount of work. However, once it is finished, the LMS can
offer abilities for advanced users to interact with the LMS in any form they
whish. For example, an instructor can implement a script using those APIs to
interact with the LMS and fully automate the process of assignment management
to satisfy with his own workflow. In addition, those set of APIs can be used
to implement a mobile device version of the \emph{Course2018} application.