%---------------------------------------------------------
% Preliminaries: Set up your own details in this file!
%----------------------------------------------------------

% Don't forget to remove the ()s in ALL of these "personalization" lines.

\title{Development of a comprehensive learning management system designed for
computer science and mathematic department with built-in automatic
code testing}

\author{Bailin He}
\dept{Computer Science}   % E.g., Physics, Computer Science,
\deptOrSchool{School}
\degree{Computer Science}  % E.g., Science, Arts, ...

\submissionMonth{July}	    % OR WHATEVER MONTH YOU ACTUALLY SUBMIT IN
\submissionYear{2018}
\copyrightYear{2018}		    % Probably the same as submissionYear.

% Use a "~" after the "r." of "Dr." so that TeX doesn't think you have
% ended a sentence (at which point it gives extra space).
\supervisor{Dr. James Diamond}

% Remove the '%' from the next line and fill in the name if desired.
%\cosupervisor{(Dr.~Your Other Supervisor)}

\headOrDirector{Dr. Darcy Benoit}
% If the head or director is an ``acting'' head or director, uncomment
% the next line (i.e., delete the '%'):
% \justActing

% You will have to ask around to find out the name of the person
% to put here... it changes from year to year.
\honoursCommittee{Dr. Matthew Lukeman} % To end of August

%-------------------------------------------------------------------------

% This outputs the title page, the approval page and the copyright page.
\firstThreePages

%-------------------------------------------------------------------------
% Now write your acknowledgements (if any).
% If you wish to acknowledge no-one, delete or comment-out the
% next few lines.

\Acknowledgments

%-------------------------------------------------------------------------

% This outputs the table of contents, lists of figures, tables, ...

\tocAndSuch

%-------------------------------------------------------------------------

\prefacesection{Abstract}

This thesis presents a learning management system (LMS) written in Python using
the Django web framework. This system is designed specifically for computer science
and mathematics instructors and students to ease the process of submitting and,
more importantly, the process of grading assignments, with a built-in code
testing module for programming assignments to build and run all students'
submitted solutions and provide informative test results for students and
markers, as well as features like syntax highlighting, Markdown input and \TeX\
notation input for mathematical equation rendering for markers and instructors
to comment on the students' assignments.


%-------------------------------------------------------------------------

% Don't mess with this line!
\afterpreface
